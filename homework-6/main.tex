% latex first.tex
% latex first.tex
% xdvi first.dvi
% dvips -o first.ps first.dvi
% gv first.ps
% lpr first.ps
% pdflatex first.tex
% acroread first.pdf
% dvipdf first.dvi first.pdf
% xpdf first.pdf
\documentclass[11pt]{article}

\usepackage{latexsym}
%\newcommand{\epsfig}{\psfig}
%\usepackage{tabularx,booktabs,multirow,delarray,array}
%\usepackage{graphicx,amssymb,amsmath,amssymb,mathrsfs}
%\usepackage{hyperref}
\usepackage{graphicx,amsmath}
\usepackage[linesnumbered, vlined, ruled]{algorithm2e}


%\usepackage[T1]{fontenc}


%\aboverulesep=0pt
%\belowrulesep=0pt

%\marginparwidth=0in
%\marginparsep=0in
\oddsidemargin=0.0in
\evensidemargin=0.0in
\headheight=0.0in
%\headsep=0in
\topmargin=-0.40in %0.35
\textheight=9.0in %9.1in
\textwidth=6.5in   %6.55in

\usepackage{fullpage}

%\setlength{\headheight}{0.2in}
%\setlength{\headsep}{0.2in}
%\setlength{\voffset}{-0.2in}

\def\report{{\em Report-Max}$(1)$}
\def\reportk{{\em Report-Max}$(k)$}

%\pagestyle{plain}

%\usepackage{listings}
%\lstloadlanguages{C, csh, make} \lstset{
%    language=C,tabsize=4,
%    keepspaces=true,
%    mathescape=true,
%    breakindent=22pt,
%    numbers=left,stepnumber=1,numberstyle=\footnotesize,
%    basicstyle=\normalsize,
%    showspaces=false,
%    flexiblecolumns=true,
%    breaklines=true, breakautoindent=true,breakindent=1em,
%    escapeinside={/*@}{@*/}
%}

%\lhead{Solution of Homework 1}
%\rhead{Haitao Wang}

\begin{document}
\baselineskip=14.0pt

\title{CS5050 \textsc{Advanced Algorithms}
\\{\Large Spring Semester, 2018}
\\ Assignment 6: Graph Algorithms I
\\ {\large \textbf{Student:} Tanner Kvarfordt - A02052217}
\\ {\large {\bf Due Date: 3:00 p.m.}, Thursday, Apr. 12, 2018 ({\bf at the beginning of CS5050 class})}}
\date{}
%\date{\today}


\maketitle
%\theoremstyle{plain}\newtheorem{theorem}{\textbf{Theorem}}

\vspace{-0.6in}

{\bf Note:} In this assignment, we assume that all input graphs are represented by adjacency lists.

\begin{enumerate}
\item

We say that a {\bf directed} graph $G$ is {\em strongly connected} if for every pair of vertices $u$ and $v$, there exists a path from $u$ to $v$ and there also exists a path from $v$ to $u$ in $G$.

Given a directed graph $G$ of $n$ vertices and $m$ edges, let $s$ be a vertex of $G$.

\begin{enumerate}
\item
Design an $O(m+n)$ time algorithm to determine whether the following is true: there exists a path from $s$ to $v$ in $G$ for all vertices $v$ of $G$. {\hfill \bf (10 points)}

\textit{Solution:} Use the BFS-shortest-path algorithm discussed in class, and then iterate through all of the vertices $w$ to verify that no $w.d$ has a value of $+\infty$. If none have a value of $+\infty$, then there exists a path from $s$ to $v$ in $G$ for all vertices $v$ of $G$. The BFS-shortest-path algorithm is $O(m + n)$. 

\item
Design an $O(m+n)$ time algorithm to determine whether the following is true: there exists a path from $v$ to $s$ in $G$ for all vertices $v$ of $G$. {\hfill \bf (10 points)}

\textit{Solution:} First note that if the graph is undirected, the BFS in item (a) is all that is needed to tell if there exists a path from $v$ to $s$ for all vertices $v$ of $G$. If the graph is directed, then temporarily reverse the direction of each edge in $G$ and then perform a BFS beginning at $s$. If the BFS visits every vertex of the temporarily reversed $G$, then there is clearly a path from every vertex $v$ of $G$ to $s$ when the edges are not reversed. It takes $m$ time to temporarily reverse edges, and $n+m$ time to perform a BFS. Therefore this algorithm is $O(n+m)$.

\item
Prove the following statement: {\em $G$ is strongly connected if and only if there exists a path in $G$ from $s$ to $v$ and there is also a path from $v$ to $s$  for all vertices $v$ of $G$}.

\textit{Solution:} In order for the definition of a strongly connected graph to be satisfied, the given statement in this problem must also be satisfied since if there is not a path from $s$ to all $v$, or not a path from all $v$ to $s$, then clearly there exists a pair of vertices $u$ and $w$ in $G$ for which there is not a path to and from each to the other. Therefore the given statement must be satisfied in order for a graph to be strongly connected. Consider a BFS of $G$ beginning at $s$. Clearly that BFS must reach every vertex $v$ of $G$ in order for $G$ to possibly be strongly connected. To seal the deal, there must also be a path from all vertices $v$ in $G$ to the vertex $s$ of $G$. If both of these conditions hold (note that these conditions make up the body of the given statement) then there clearly is a path from any vertex of $G$ to any other vertex of $G$ via $s$, therefore making the graph strongly connected.

{\hfill \bf (10 points)}

{\bf Remark:} According to the above statement, we can determine whether $G$ is strongly connected in $O(m+n)$ time by using your algorithms for the above two questions (a) and (b).

\end{enumerate}


\item
Given an undirected graph $G$ of $n$ vertices and $m$ edges, suppose $s$ and $t$ are two vertices in $G$. We have already known that we can find a shortest path from $s$ to $t$ by using the breadth-first-search (BFS) algorithm. However, there might be multiple different shortest paths from $s$ to $t$ (e.g., see Fig.~\ref{fig:undirectedpaths} as an example).

Design an $O(m+n)$ time algorithm to compute the number of different shortest paths from $s$ to $t$ (your algorithm does not need to list all the shortest paths, but only report their number). (Hint: Modify the BFS algorithm.)
{\hfill \bf (20 points)}

\textit{Solution:} Assuming that
\[v.color = 
\begin{cases} 
white & \mbox{if v is not visited} \\ 
blue & \mbox{if v is visited but its adjacent vertices have not been checked} \\
red & \mbox{if v is visited and its adjacent vertices have been checked}
\end{cases}\],
perform the BFS algorithm, but allow all vertices that are not colored red to be enqueued. Once $t$ is reached by the shortest path $p$, begin a counter of paths that have the same length as $p$ (including $p$). After reaching $t$, continue to dequeue the remaining vertices and check if each one is $t$, incrementing the counter as appropriate. If $v.d$ of any of the remaining vertices is less than the length of $p$, continue to enqueue that vertex's neighbors. Once the queue is empty, return the counter.
The pseudocode below has been shortened by using a level-order traversal of the graph, and so doesn't match this description exactly. However, the principle is the same.
\newpage
\begin{verbatim}
BFS-Shortest-Path-Counter(G, s, t)
    for each vertex v in G
        v.color = white
    s.color = blue
    enqueue(Q, s)
    count = 0     // variable to hold the overall result
    found = false // variable indicating whether or not t has been reached
    while Q is not empty
        currentLevel = Q.size
            for i = 1 to currentLevel
                u = dequeue(Q)
                if u == t
                    ++count
                    found = true
                for v in adj(u) // for each vertex in u's adjacency list
                    if v == t
                        found = true
                        ++count
                    if v.color != red and !found
                        v.color = blue
                        enqueue(Q, v)
                u.color = red
            if found
                return count
\end{verbatim}

This algorithm is still just a BFS of the graph $G$, and hence is $O(m + n)$ when using an adjacency list.

%\begin{figure}[h]
%\begin{minipage}[t]{\linewidth}
%\begin{center}
%\includegraphics[totalheight=1.2in]{undirectedpaths.eps}
%\caption{There are three different shortest paths from $s$ to $t$: $(s,a,b,t),(s,c,b,t), (s,c,e,t)$. }\label{fig:undirectedpaths}
%\end{center}
%\end{minipage}
%\end{figure}

{\bf Note:} In case you are interested, the following is an application of the above problem in social networks. 

A number of art museums around the country have been featuring work
by an artist named Mark Lombardi (1951--2000), consisting of a set of
intricately rendered graphs. Building on a great deal of research, these
graphs encode the relationships among people involved in major political
scandals over the past several decades: the vertices correspond to participants,
and each edge indicates some type of relationship between a pair
of participants. And so, if you peer closely enough at the drawings, you
can trace out ominous-looking paths from a high-ranking government
official, to a former business partner, to a bank in Switzerland, to
a shadowy arms dealer.

Such pictures form striking examples of {\em social networks}, which
have vertices representing people and organizations,
and edges representing relationships of various kinds. And the
short paths that abound in these networks have attracted considerable
attention recently, as people ponder what they mean. In the case of Mark
Lombardi¡¯s graphs, they hint at the short set of steps that can carry you
from the reputable to the disreputable.


Of course, a single, spurious short path between vertices $s$ and $t$ in
such a network may be more coincidental than anything else; a large
number of short paths between $s$ and $t$ can be much more convincing.
So in addition to the problem of computing a single shortest $s$-$t$ path
in a graph $G$, social networks researchers have looked at the problem of
determining the number of shortest $s$-$t$ paths, which is the problem we are now considering.

\item
Given a directed-acyclic-graph (DAG) $G$ of $n$ vertices and $m$ edges,
let $s$ and $t$ be two vertices of $G$. There might be multiple different paths (not necessarily shortest paths) from $s$ to $t$ (e.g., see Fig.~\ref{fig:paths} for an example).
Design an $O(m+n)$ time algorithm to compute the number of different paths in $G$ from $s$ to $t$. {\hfill \bf (20 points)}

\textit{Solution:} For $v \in G$, let $v.c$ define the number of paths from $v$ to $t$. The solution to this algorithm lies in the fact that the number of paths from a vertex $v$ to $t$ is the sum of the number of paths from each child of $v$ to $t$ where $t$ is considered to have one path to itself. That is, $v.c = \sum_{u \in adj(v)} u.c$ where $t.c = 1$. Since the answer depends on ancestors of $s$, we'll need to perform a topological sort and work backwards from $t$ in order to ensure that all descendants' $.c$ values are computed before we try to access them. Therefore the pseudocode for this algorithm is as follows:
\begin{verbatim}
DAG-Paths(G, s, t)
    for v in G
        v.c = 0
    t.c = 1
    A = array containing topological sort of G
    reverse A
    t_index := an integer giving the location of t in A
    for i = 0 to n - 1
        if A[i] == t
            t_index = i
            break
    for i from t_index + 1 to n - 1
        u = A[i]
        for v in adj(u)
            u.c += v.c
    return s.c
\end{verbatim}
Since a topological sort takes $m+n$ time, the algorithm as a whole takes $n + (m+n) + n + n + m = 3n + 2m = O(m + n)$ time.

%\begin{figure}[h]
%\begin{minipage}[t]{\linewidth}
%\begin{center}
%\includegraphics[totalheight=1.0in]{paths.eps}
%\caption{\footnotesize There are three different paths from $s$ to $t$: $s\rightarrow t$, $s\rightarrow b\rightarrow t$, and $s\rightarrow a\rightarrow b\rightarrow t$.} \label{fig:paths}
%\end{center}
%\end{minipage}
%\vspace*{-0.10in}
%\end{figure}

\item
In class we have studied an algorithm for computing shortest paths in DAGs. Particularly, the {\em length} of a path is defined to be the total sum of the weights of all edges in the path.

Sometimes we are not only interested in a shortest path but also care about the number of edges of the path. For example, consider a flight graph in which each vertex is an airport and each edge represents a flight segment from one airport to another, and the weight of every edge represents the price of that flight segment. If you want to fly from airport $s$ to another one $t$, you are certainly interested in a shortest path from $s$ to $t$ to minimize the total price you have to pay. But you probably do not want a path with too many stops (i.e., too many edges) even if it is relatively cheap. For example, suppose you can only accept a path with at most two stops (i.e., three edges). Then, your goal is essentially to find a shortest path from $s$ to $t$ such that the path has at most three edges. This is the problem we are considering in this exercise.

Let $G$ be a DAG (directed-acyclic-graph) of $n$ vertices and $m$ edges. Each edge $(u,v)$ of $E$ has a weight $w(u,v)$ (which can be positive, zero, or negative). Let $k$ be a positive integer, which is given in the input. A path in $G$ is called a {\em $k$-link} path if the path has {\bf no more than} $k$ edges. Let $s$ and $t$ be two vertices of $G$. A {\em $k$-link shortest path} from $s$ to $t$ is defined as a $k$-link path from $s$ to $t$ that has the minimum total sum of edge weights among all possible $k$-link $s$-to-$t$ paths in $G$. Refer to Fig.~\ref{fig:klinkpath} for an example.

Design an $O(k(m+n))$ time algorithm to compute a $k$-link shortest path from $s$ to $t$. For simplicity, you only need to return the length of the path and do not need to report the actual path. (Hint: use dynamic programming and modify the shortest path algorithm on DAG we discussed in class.)
{\hfill \bf (20 points)}

\textit{Solution:} This algorithm will be very similar to the original DAG-shortest-path algorithm. In order to compute the $k$-link shortest path, we will compute all $k$-link shortest paths from 0 to $k$ between the two vertices, and then choose the minimum. To do this, we will need to augment each node with an attribute $k$. For each node $v$ in $G$, $v.k$ is the number of links in the path and $v.d$ is the shortest weighted distance between $s$ and $v$. The pseudocode is as follows:
\begin{verbatim}
for each vertex v in G
    v.d = infinity
    v.k = 0
s.d = 0
topologically sort G
F[0...k] := Array from 0 to k storing the best answer for k = index of F
for i from 0 to k
    for each vertex u in the topological order
        for each vertex v in adj(u)
            if v.d > u.d + w(u,v) and v.k+1 <= k
                v.d = u.d
                v.k = k + 1
    F[i] = t.d
    for each vertex v in G
        v.d = infinity
        v.k = 0
    s.d = 0
return min{F[0...k]}
\end{verbatim}
By examining the for-loops, we can see that this algorithm runs in $k(m+n+n)=k(2n+m)=O(k(m+n))$.

%\begin{figure}[h]
%\begin{minipage}[t]{\linewidth}
%\begin{center}
%\includegraphics[totalheight=1.0in]{klinkpath.eps}
%\caption{\footnotesize Illustrating a DAG: the number besides each edge is the weight of the edge. If we are looking for a $2$-link (i.e., $k=2$) shortest path from $s$ to $t$, then it is the path consisting of the only edge $(s,t)$, whose length is $5$. Although the path $s\rightarrow a\rightarrow b\rightarrow t$ is shorter, it is not a $2$-link path since it has three edges. Note that the path with the only edge $(s,t)$ is a $2$-link path because it has one edge, which is indeed {\bf no more than} two edges.} \label{fig:klinkpath}
%\end{center}
%\end{minipage}
%\vspace*{-0.10in}
%\end{figure}

\end{enumerate}


{\bf Total Points:} 90
\end{document}

%\begin{figure}[h]
%\begin{minipage}[t]{\linewidth}
%\begin{center}
%\includegraphics[totalheight=2in]{figexample.eps}
%\caption{An example}\label{fig:example}
%\end{center}
%\end{minipage}
%\end{figure}

%\begin{table}[h]
%\begin{center}
%\begin{tabularx|tabular}{0.75\textwidth}{|p{1.75cm}||X|X|X|X|X|X|X|X|X|X|}%{|l||c|c|c|c|c|c|c|c|c|c|}
%\hline
%Capacity & 7 &8&9&6&5&4&7&4&10&9\\
%\hline
%Allocation & 7&7&7&0&4&4&4&0&9&9\\
%\hline
%\end{tabularx|tabular}
%\end{center}
%\label{tab:tabexample}
%\end{table}


%\begin{description}
%\item[Algorithm Description]
%\item[Pseudocode for the Algorithm]
%\item[Analysis of Running time]
%\item[Argument for Correctness]
%\end{description}

%\begin{thebibliography}{99}
%\bibitem{ref:clrs}
%Thomas~H.~Cormen, Charles~E.~Leiserson, Ronald~E.~Rivest,
%Clifford~Stein.
%\newblock{Introduction to Algorithms, Second Edition, pages 568-570,}
%\newblock{\emph{The MIT Press}, 2005.}
%\end{thebibliography}

%\begin{figure}[t]
%\begin{minipage}[t]{0.49\linewidth}
%\begin{center}
%\includegraphics[totalheight=0.9in]{visibilitymap.eps}
%\caption{\footnotesize Illustrating the horizontal visibility map of
%a splinegon.} \label{fig:visibilitymap}
%\end{center}
%\end{minipage}
%\hspace*{0.02in}
%\begin{minipage}[t]{0.49\linewidth}
%\begin{center}
%\includegraphics[totalheight=0.9in]{trapezoid.eps}
%\caption{\footnotesize Illustrating the decomposition of a
%trapezoid: (a) The black points on the base $cd$ are vertices; (b) a
%decomposition of the trapezoid.} \label{fig:trapezoid}
%\end{center}
%\end{minipage}
%\vspace*{-0.15in}
%\end{figure}

