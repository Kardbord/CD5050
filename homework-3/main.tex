% latex first.tex
% latex first.tex
% xdvi first.dvi
% dvips -o first.ps first.dvi
% gv first.ps
% lpr first.ps
% pdflatex first.tex
% acroread first.pdf
% dvipdf first.dvi first.pdf
% xpdf first.pdf
\documentclass[11pt]{article}

\usepackage{latexsym}
%\newcommand{\epsfig}{\psfig}
%\usepackage{tabularx,booktabs,multirow,delarray,array}
%\usepackage{graphicx,amssymb,amsmath,amssymb,mathrsfs}
%\usepackage{hyperref}
\usepackage{graphicx,fancyhdr,amsmath}
\usepackage[linesnumbered, vlined, ruled]{algorithm2e}


%\usepackage[T1]{fontenc}


%\aboverulesep=0pt
%\belowrulesep=0pt

%\marginparwidth=0in
%\marginparsep=0in
\oddsidemargin=0.0in
\evensidemargin=0.0in
\headheight=0.0in
%\headsep=0in
\topmargin=-0.40in %0.35
\textheight=9.0in %9.1in
\textwidth=6.5in   %6.55in

\usepackage{fullpage}

%\setlength{\headheight}{0.2in}
%\setlength{\headsep}{0.2in}
%\setlength{\voffset}{-0.2in}


%\pagestyle{plain}

%\usepackage{listings}
%\lstloadlanguages{C, csh, make} \lstset{
%    language=C,tabsize=4,
%    keepspaces=true,
%    mathescape=true,
%    breakindent=22pt,
%    numbers=left,stepnumber=1,numberstyle=\footnotesize,
%    basicstyle=\normalsize,
%    showspaces=false,
%    flexiblecolumns=true,
%    breaklines=true, breakautoindent=true,breakindent=1em,
%    escapeinside={/*@}{@*/}
%}

%\lhead{Solution of Homework 1}
%\rhead{Haitao Wang}

\begin{document}

\baselineskip=14.0pt

\title{CS5050 \textsc{Advanced Algorithms}
\\{\Large Spring Semester, 2018}
\\ Assignment 3: Prune and Search
\\ {\large \textbf{Student:} Tanner Kvarfordt - A02052217}
\\ {\large {\bf Due Date: 3:00 p.m.}, Thursday, Feb. 22, 2018 ({\bf at the beginning of CS5050 class})}}
\date{}
%\date{\today}

\maketitle
%\theoremstyle{plain}\newtheorem{theorem}{\textbf{Theorem}}

\vspace{-0.7in}


\begin{enumerate}

\item
{\bf (20 points)}
Suppose you are given an array $A$ with $n$ entries, with each entry
holding a distinct number (i.e., no two numbers of $A$ are
		equal). You are told that the sequence of
		values $A[1], A[2], \ldots, A[n]$ is {\em unimodal:} For some
		index $p$ between $1$ and $n$, the values in the array entries
		increase up to position $p$ in $A$ and then decrease the
		remainder of the way until position $n$. (So if you were to
		draw a plot with the array position $j$ on the $x$-axis and
		the value of the entry $A[j]$ on the $y$-axis, the plotted
		points would rise until $x$-value $p$, where they'd achieve
		their maximum, and then fall from there on.)


You'd like to find the ``peak entry'' $p$ without having to read the entire array -- in fact, by reading as few entries of $A$ as possible. Show how to find
the entry $p$ by reading at most $O(\log n)$ entries of $A$. In other words, design an $O(\log n)$ time algorithm to find the peak entry $p$.

For example, let $A=\{1, 4, 6, 8, 11, 12, 10, 9, 7\}$, which is a unimodal array. The peak entry of $A$ is $12$. So the output of your algorithm is either 12 or the index of 12 in $A$.

\textit{Solution:} Perform a modified binary search by finding the mid and peaking left and right. If numbers to the left and right are both smaller, we have located the peak entry. If the number to the left is smaller but the number to the right is larger, then the peak entry is to the right. If number to the left is larger and the number to the right is smaller, then the peak entry is to the left.
\begin{verbatim}
// U is a unimodal array
peakEntry(U[1...n]) {
    if (U.size == 1) return U[1]
    mid = floor(n/2)
    if (U[mid] > U[mid-1] and U[mid] > U[mid+1])
        return U[mid]
    if (U[mid] < U[mid-1])
        return peakEntry(U[1...mid])
    if (U[mid] < U[mid+1])
        return peakEntry(U[mid...n])
}
\end{verbatim}
This is a modified binary search algorithm, meaning we look at $\log n$ of the entries of the array at most. Therefore this algorithm is $O(\log n)$.

\item
{\bf (20 points)}
In the SELECTION algorithm we studied in class, the input numbers are divided into groups of five. Will the algorithm still work in linear time if they are divided into groups of seven? Please justify your answer.

\textit{Solution:} The algorithm will still work in linear time. If the input numbers are divided into groups of seven, the recurrence becomes as follows:
\[T(n) = T\left(\frac{n}{7}\right) + T\left(\frac{5}{7}n\right) + n\]
Using the guess and verification method, we can see that the recurrence is still $O(n)$.
\begin{gather}
\text{Assume: } T(n) = O(n) \leq cn \\
 T\left(\frac{n}{7}\right) = O(n) \leq \frac{cn}{7} \\
 T\left(\frac{5}{7}n\right) = O(n) \leq \frac{5cn}{7} \\
 \Rightarrow \frac{cn}{7} + \frac{5cn}{7} + n \leq cn \\
 \Rightarrow \frac{6cn}{7} + n \leq cn \\
 \Rightarrow n \leq \frac{cn}{7} \text{ for } c \geq 7
\end{gather}

\item
{\bf (20 points)}
Suppose you are consulting for an oil company, which is planning a
large pipeline (called the {\em main pipeline}) running horizontally from east to west through
an oil field of $n$ wells. From each well, a {\em spur pipeline} is to
be connected directly to the main pipeline along a shortest path
(going to either the north or the south).

Suppose there are $n$ wells, represented by $n$ points $p_1,p_2,\ldots,p_n$ in the plane.
We are given the $x$- and $y$-coordinates of the $n$ wells $p_i=(x_i,y_i)$ for $i=1,2,\ldots,n$. Note that the wells are not given in any sorted order. Our goal is to pick an optimal location for the main pipeline (i.e., find the $y$-coordinate of the main pipeline) such that the
{\bf total sum of the lengths} of the spur pipelines is minimized. For simplicity, we assume no two wells have the same $x$-coordinate or $y$-coordinate.

Design an $O(n)$ time algorithm to compute an optimal location for the
main pipeline.

\textit{Solution:} In order to minimize the total sum of the lengths of the spur pipelines, the main pipeline should run as closely as possible to each well. Therefore the optimal location for the main pipeline is directly through the well with the 
median\footnote{median: denoting or relating to a value or quantity lying at the midpoint of a frequency distribution of observed values or quantities} $y$ coordinate. The selection algorithm (discussed in class) can be used to locate the median $y$ coordinate in $O(n)$ time. Note that in the event of an even number of wells, the well on either side of where the median would be will also produce the shortest overall length of spur pipeline needed. To see why this is, consider solving this problem for two wells with unique $x$- and $y$-coordinates. Let the vertical ($y$-coordinate) distance between the two be denoted by $d$. Clearly, the main pipeline must be placed somewhere between the two wells or through one of the wells. Regardless of where in that spectrum the main line is placed, a total length of $d$ spur pipeline will be needed to connect each well to the main line regardless of where that main line runs. In summary, the solution is to run the main pipeline through the well with the median $y$ coordinate. The selection algorithm can be used to locate the well with the median $y$ coordinate in $O(n)$ time.

%\begin{figure}[h]
%\begin{minipage}[t]{\linewidth}
%\begin{center}
%\includegraphics[totalheight=2.0in]{wells.eps}
%\caption{Illustrating Problem 3: the horizontal solid line is the main pipeline and the dashed vertical segments are the spur pipelines.}\label{fig:wells}
%\end{center}
%\end{minipage}
%\end{figure}



\item
{\bf (30 points)}
Here is a generalized version of the selection problem, called {\em multiple selection}. Let $A[1\cdots n]$ be an array of $n$ numbers. Given a sequence of $m$ sorted integers $k_1, k_2,\ldots, k_m$, with $1\leq k_1<k_2<\cdots<k_m\leq n$, the {\em multiple selection problem} is to find the $k_i$-th smallest number in $A$ for all $i=1,2,\ldots,m$. For simplicity, we assume that no two numbers of $A$ are equal.

For example, let $A=\{1, 5, 9, 3, 7, 12, 15, 8, 21\}$, and $m=3$ with
$k_1 = 2$, $k_2 = 5$, and $k_3=7$. Hence, the goal is to find the 2nd,
the $5$-th, and the $7$-th smallest numbers of $A$, which are $3$,
$8$, and $12$, respectively.

\begin{enumerate}
\item
Design an $O(n\log n)$ time algorithm for the problem. {\hfill \bf (5 points)} \\
\textit{Solution:} Perform merge-sort on $A$. Then the $k_i$-th smallest number is simply $A[k_i]$. Since merge-sort takes $O(n\log n)$ time, and returning $A[k_i]$ $m$ times takes $O(m)$ time, this algorithm takes $n\log n + m = O(n \log n)$ time.

\item
Design an $O(nm)$ time algorithm for the problem. Note that this is better than the $O(n\log n)$ time algorithm if $m<\log n$. {\hfill \bf (5 points)} \\
\textit{Solution:} Perform the single selection algorithm (which we discussed in class) $m$ times, once for each $k$-th smallest number we are trying to find. Since the selection algorithm takes $O(n)$ time, and we are performing it $m$ times, this algorithm is $O(mn)$.

\item
Improve your algorithm to $O(n\log m)$ time, which is better than both the $O(n\log n)$ time and the $O(nm)$ time algorithms. {\hfill \bf (20 points)}

\textit{Solution:} This algorithm will be similar to the single selection algorithm. Instead of passing a single $k$-th smallest value to find, pass an array, $K$, of $k$-th smallest values to find. For simplicity in the pseudo-code below, a result set, $R$, is passed in by reference. If $K$ is empty, return from the function. Otherwise, begin by computing the $K$[mid]-th smallest value (we'll call it $y$) using the single selection algorithm and add it to the solution set. That is, if $m=3$ compute the $k_2$-th smallest value of $A$. Next, partition $A$ into two sub-arrays $A_1$ and $A_2$, where $A_1$ contains the elements of $A$ smaller than $y$ and $A_2$ contains the elements of $A$ larger than $y$. Also partition $K$ into two sub-arrays $K_1$ and $K_2$, where $K_1$ contains values smaller than $K$[mid] and $K_2$ contains values larger than $K$[mid]. For every $k$ in $K_2$, subtract $K$[mid] from $k$. Finally, compute the remaining $k$-th smallest numbers by making two recursive calls to this algorithm: one using $K_1$ and $A_1$, and one using $K_2$ and $A_2$. Pseudo-code and complexity analysis for this algorithm are below.
\pagebreak
\begin{verbatim}
// A is the array of numbers
// K is the array of k-th smallest numbers to find
// R is the result set, passed by reference
multiselect(A, K, R) {
    if (K.isempty) return;
    
    mid = floor(K.size / 2)
    y = selection(A, K[mid]) // find the median k-smallest value in A
    add y to R // add the computed k-th smallest value to the solution set

    // Reduce the problem size
    A_1 = (elements of A) < y
    A_2 = (elements of A) > y

    i = K[mid] // note this is equivalent to i = A.size + 1
    K_1 = (elements of K) < i
    K_2 = (elements of K) > i

    // Recursively compute remaining results
    multiselect(A_1, K_1, R)
    
    for each k in K_2
        k = k - i
        
    multiselect(A_2, K_2, R)
}
\end{verbatim}

\textbf{Complexity Analysis:} 
\begin{enumerate}
\item this algorithm has a recurrence denoted by $T(n,m)$
\item selection algorithm takes $O(n)$ time
\item partitioning $A$ takes $n$ time
\item partitioning $K$ takes $m$ time
\item Two recursive calls with worst-possible partition takes $T\left(n-1,\frac{m}{2}\right)$ + $T\left(1,\frac{m}{2}\right)$
\end{enumerate}
Therefore the recurrence for this function is 
$T(n,m) = 2n + m + T\left(n-1,\frac{m}{2}\right)$ + $T\left(1,\frac{m}{2}\right)$.
Using guess and verification, it can be shown that this algorithm is $O(n\log m)$.\\
Assume the following:
\begin{align*}
T(n,m)                        &= O(n\log m) \leq cn\log m\\
T\left(n-1,\frac{m}{2}\right) &= O(n\log m) \leq c(n-1)\log m\\
T\left(1,\frac{m}{2}\right)   &= O(n\log m) \leq c\log m
\end{align*}

It then follows that
\[2n + m + c(n-1)\log (m/2) + c\log(m/2) \leq cn\log m\]
\[\Rightarrow 2n + m \leq c(n\log m - (n - 1)\log (m/2) - \log (m/2))\]
\[\Rightarrow 2n + m \leq c(n\log m - n\log (m/2) + \log (m/2) - \log (m/2))\]
\[2n + m \leq c(n \log m - n \log (m/2))\]
Which holds for $c \geq \frac{2n + m}{n \log m - n \log (m/2)}$. Therefore this algorithm is $O(n\log m)$.

\end{enumerate}

\end{enumerate}

{\bf Total Points:} 90

\end{document}

%\begin{figure}[h]
%\begin{minipage}[t]{\linewidth}
%\begin{center}
%\includegraphics[totalheight=2in]{figexample.eps}
%\caption{An example}\label{fig:example}
%\end{center}
%\end{minipage}
%\end{figure}

%\begin{table}[h]
%\begin{center}
%\begin{tabularx|tabular}{0.75\textwidth}{|p{1.75cm}||X|X|X|X|X|X|X|X|X|X|}%{|l||c|c|c|c|c|c|c|c|c|c|}
%\hline
%Capacity & 7 &8&9&6&5&4&7&4&10&9\\
%\hline
%Allocation & 7&7&7&0&4&4&4&0&9&9\\
%\hline
%\end{tabularx|tabular}
%\end{center}
%\label{tab:tabexample}
%\end{table}


%\begin{description}
%\item[Algorithm Description]
%\item[Pseudocode for the Algorithm]
%\item[Analysis of Running time]
%\item[Argument for Correctness]
%\end{description}

%\begin{thebibliography}{99}
%\bibitem{ref:clrs}
%Thomas~H.~Cormen, Charles~E.~Leiserson, Ronald~E.~Rivest,
%Clifford~Stein.
%\newblock{Introduction to Algorithms, Second Edition, pages 568-570,}
%\newblock{\emph{The MIT Press}, 2005.}
%\end{thebibliography}

%\begin{figure}[t]
%\begin{minipage}[t]{0.49\linewidth}
%\begin{center}
%\includegraphics[totalheight=0.9in]{visibilitymap.eps}
%\caption{\footnotesize Illustrating the horizontal visibility map of
%a splinegon.} \label{fig:visibilitymap}
%\end{center}
%\end{minipage}
%\hspace*{0.02in}
%\begin{minipage}[t]{0.49\linewidth}
%\begin{center}
%\includegraphics[totalheight=0.9in]{trapezoid.eps}
%\caption{\footnotesize Illustrating the decomposition of a
%trapezoid: (a) The black points on the base $cd$ are vertices; (b) a
%decomposition of the trapezoid.} \label{fig:trapezoid}
%\end{center}
%\end{minipage}
%\vspace*{-0.15in}
%\end{figure}

