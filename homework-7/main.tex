% latex first.tex
% latex first.tex
% xdvi first.dvi
% dvips -o first.ps first.dvi
% gv first.ps
% lpr first.ps
% pdflatex first.tex
% acroread first.pdf
% dvipdf first.dvi first.pdf
% xpdf first.pdf
\documentclass[11pt]{article}

\usepackage{latexsym}
%\newcommand{\epsfig}{\psfig}
%\usepackage{tabularx,booktabs,multirow,delarray,array}
%\usepackage{graphicx,amssymb,amsmath,amssymb,mathrsfs}
%\usepackage{hyperref}
\usepackage{graphicx}
\usepackage[linesnumbered, vlined, ruled]{algorithm2e}


%\usepackage[T1]{fontenc}


%\aboverulesep=0pt
%\belowrulesep=0pt

%\marginparwidth=0in
%\marginparsep=0in
\oddsidemargin=0.0in
\evensidemargin=0.0in
\headheight=0.0in
%\headsep=0in
\topmargin=-0.40in %0.35
\textheight=9.0in %9.1in
\textwidth=6.5in   %6.55in

\usepackage{fullpage}

%\setlength{\headheight}{0.2in}
%\setlength{\headsep}{0.2in}
%\setlength{\voffset}{-0.2in}

\def\report{{\em Report-Max}$(1)$}
\def\reportk{{\em Report-Max}$(k)$}

%\pagestyle{plain}

%\usepackage{listings}
%\lstloadlanguages{C, csh, make} \lstset{
%    language=C,tabsize=4,
%    keepspaces=true,
%    mathescape=true,
%    breakindent=22pt,
%    numbers=left,stepnumber=1,numberstyle=\footnotesize,
%    basicstyle=\normalsize,
%    showspaces=false,
%    flexiblecolumns=true,
%    breaklines=true, breakautoindent=true,breakindent=1em,
%    escapeinside={/*@}{@*/}
%}

%\lhead{Solution of Homework 1}
%\rhead{Haitao Wang}

\begin{document}
\baselineskip=14.0pt

\title{CS5050 \textsc{Advanced Algorithms}
\\{\Large Spring Semester, 2018}
\\ Assignment 7: Graph Algorithms II
\\ {\large \textbf{Student:} Tanner Kvarfordt - A02052217}
\\ {\large {\bf Due Date: 3:00 p.m.}, Tuesday, Apr. 24, 2018 ({\bf at the beginning of CS5050 class})}}
\date{}
%\date{\today}


\maketitle
%\theoremstyle{plain}\newtheorem{theorem}{\textbf{Theorem}}

\vspace{-0.6in}

{\bf Note:} In this assignment, we assume all input graphs are represented by adjacency lists.

\begin{enumerate}

\item
{\bf (20 points)}
Given a directed graph $G$ of $n$ vertices and $m$ edges, each edge $(u,v)$ has a weight $w(u,v)$, which can be positive, zero, or negative.
The {\em bottleneck-weight} of any path in $G$ is defined to be the
{\bf largest} weight of all edges in the path. Let $s$ and $t$ be two vertices of $G$. A {\em minimum bottleneck-weight path} from $s$ to $t$ is a path with the smallest bottleneck-weight among all paths from $s$ to $t$ in $G$.

Modify Dijkstra's algorithm to compute a  minimum bottleneck-weight path from $s$ to $t$. Your algorithm should have the same time complexity as Dijkstra's algorithm.

\textit{Solution:} This algorithm will be very similar to Dijkstra's Algorithm. For all $v \in G$, let $v.b$ represent the smallest bottleneck weight of a path containing $v$. Initially, $v.b$ will be $\infty$, except for $s$, where $s.b = -\infty$. Put all vertices in a set, $Q$, then repeatedly find a vertex $u \in Q$. whose $u.b$ value is the smallest and update $v.b$ for each outgoing edge 
$(u, v)$. Also, for each $v \in G$, let $v.pre$ indicate the preceding node in the path. The pseudocode is as follows:
\begin{verbatim}
Min-Bottleneck(G, s) {
    for v in G
        v.b = positive infinity
    s.b = negative infinity
    Q: a set of all vertices
    while Q not empty
        u = extractMin(Q)
        for v in adj(u)
            if u.b < w(u,v)
                if v.b == positive infinity
                    v.b = max(w(u,v), u.b)
                else v.b = min(v.b, max(w(u,v), u.b))
                v.pre = u
                decreaseKey(Q, v, v.b)
}
\end{verbatim}
In order to output the min-bottleneck path after the algorithm has run, simply trace back from the desired vertex using its .pre attribute. Once $s$ is reached, reverse the path to get it in proper order from $s$ to destination node. This takes at most $O(n)$ time, so this algorithm has the same complexity as Dijkstra's Algorithm: $O((m+n)\log_2n)$.

\item
Let $G=(V,E)$ be an undirected connected graph, and each edge $(u,v)$ has a positive weight $w(u,v)>0$. Let $s$ and $t$ be two vertices of $G$. Let $\pi(s,t)$ denote a shortest path from $s$ to $t$ in $G$. Let $T$ be a minimum spanning tree of $G$. Please answer the following questions and explain why you obtain your answers.

\begin{enumerate}
\item
Suppose we increase the weight of every edge of $G$ by a positive value $\delta>0$. Then, is $\pi(s,t)$ still a shortest path from $s$ to $t$?
{\hfill \bf (10 points)}

\textit{Solution:} Not necessarily. Consider a graph with the following paths: $a,b,c,d,e,f$ where each edge in the path has weight $1$, and $a,f$, where the single edge has a weight of $10$. Clearly, the shortest path from $a$ to $f$ is $a,b,c,d,e,f$, with a total weight of $5$. Now imagine we increase the weight of each edge by $\delta=5$. Now the path $a,b,c,d,e,f$ has a total weight of $25$, while the path $a,f$ now has weight of $15$, making it the new shortest path from $a$ to $f$.

\item
Suppose we increase the weight of every edge of $G$ by a positive value $\delta>0$. Then, is $T$ still a minimum spanning tree of $G$?
{\hfill \bf (10 points)}

\textit{Solution:} Yes. The reason for part (a) not being true is that there may be several paths from $s$ to $t$, each with a different amount of edges. However, a minimum spanning tree always has $n-1$ edges, and so modifying each edge by any positive value of $\delta$ will maintain the same set of minimum spanning trees.

\end{enumerate}

\item
{\bf (20 points)}
Let $G=(V,E)$ be an undirected connected graph of $n$ vertices and $m$ edges. Suppose each edge of $G$ has a color of either {\em blue} or {\em red}. Design an algorithm to
find a spanning tree $T$ of $G$ such that $T$ has as few red edges as possible. Your algorithm should run in $O((n+m)\log n)$ time. 

\textit{Solution:} Kruskal's algorithm can be easily modified to solve this problem. Let $T$ be the minimum spanning tree. Initially $T=\emptyset$. Sort all edges by color, where blue comes before red. Then, for each edge $(u,v)$ in the sorted order, if $T \cup\{(u,v)\}$ produces no cycles (use union-find to check for cycles), then $T=T\cup \{(u,v)\}$. The pseudocode is as follows:
\begin{verbatim}
Kruskal-Mod(G) {
    W = set of edges in G sorted by color, with blue first // O(mlog(m))
    T = empty set of edges // spanning tree
    for each edge (u,v) in W
        if union(T, (u,v)) produces no cycles // union-find is O(log(n))
            T = union(T, (u,v))
    return T
}
\end{verbatim}
This algorithm is just Kruskal's algorithm, but with edges sorted by color, so the complexity is $O(m\log_2n)$, which is better than $O((n+m)\log n)$ since $m < (n+m)$.
\end{enumerate}
{\bf Total Points: 60}
\end{document}

