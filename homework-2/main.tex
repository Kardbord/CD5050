\documentclass[11pt]{article}

\usepackage{latexsym}
%\newcommand{\epsfig}{\psfig}
\usepackage{tabularx,booktabs,multirow,delarray,array}
\usepackage{graphicx,amssymb,amsmath,amssymb,mathrsfs}
%\usepackage{hyperref}
\usepackage[linesnumbered, vlined, ruled]{algorithm2e}


%\usepackage[T1]{fontenc}


\aboverulesep=0pt
\belowrulesep=0pt

%\marginparwidth=0in
%\marginparsep=0in
\oddsidemargin=0.0in
\evensidemargin=0.0in
\headheight=0.0in
%\headsep=0in
\topmargin=-0.40in %0.35
\textheight=9.0in %9.1in
\textwidth=6.5in   %6.55in

\usepackage{fullpage}


\begin{document}
\baselineskip=14.0pt

\title{CS5050 \textsc{Advanced Algorithms}
\\{\Large Spring Semester, 2018}
\\ Assignment 2: Divide and Conquer
\\ {\large {\bf Due Date: 3:00 p.m.}, Thursday, Feb. 8, 2018 ({\bf at the beginning of CS5050 class})}}
\date{}
%\date{\today}


\maketitle
%\theoremstyle{plain}\newtheorem{theorem}{\textbf{Theorem}}

\vspace{-0.6in}

\noindent
{\bf Note:} The assignment is much more difficult than Assignment 1, so please start early. As for the last question of Assignment 1, for each of the algorithm design problems in all assignments of this class, you are required to clearly describe the main idea of your algorithm. Although it is not required, you are encouraged to give the pseudo-code (unless you feel it is really not necessary). You also need to briefly explain why your algorithm is correct if the correctness is not that obvious. Finally, please analyze the running time of your algorithm.


\begin{enumerate}



\item
Let $A[1\cdots n]$ be an array of $n$ {\em distinct} numbers (i.e., no two
numbers are equal). If $i<j$ and $A[i]>A[j]$, then the pair
$(A[i],A[j])$ is called an {\em inversion} of $A$.

\begin{enumerate}
\item
	List all inversions of the array $\{ 4,2,9,1,7\}$. {\hfill \bf (5 points)}
	
    \textit{Answer:} Inversions: $\{(4,2),(4,1),(2,1),(9,1),(9,7)\}$
    
\item
	What array with elements from the set $\{1,2,\ldots,n\}$ has the
	most inversions? How many inversions does it have?
	{\hfill \bf (5 points)}
    
    \textit{Answer:} The array $\{n, n - 1,\ldots,2,1\}$ has the most inversions. It has
    \[(n-1)+(n-2)+(n-3)+\cdots+(n-n+1)+(n-n) = \sum_{i=1}^n(n-i) = \frac{(n-1)n}{2}\] inversions.

\item
	Give a divide-and-conquer algorithm that computes the number of
	inversions in array $A$ in $O(n\log n)$
	time. ({\bf Hint:} Modify merge sort.)	{\hfill \bf (20 points)}
    
    \textit{Answer:} Perform merge sort (sorting from smallest to largest). However, during the merge phase of merge sort, we'll do things 
    a little differently. Each time an element is merged, check to see how many elements are to its right in the new sublist. This number
    represents the number of inversions for that element. Keep a running total of these inversions and return it. An example of the
    modified merge is provided below, where underlines represent sublists, the number in parenthesis represents the number of inversions
    detected on that line, and the circled numbers represent those inversions.
    
    \begin{center}
    \begin{tabular}{cccccc}
      \underline{5} & \underline{4} & \underline{3} & \underline{2} & \underline{1} & (0) \\
      \underline{5} & \multicolumn{2}{c}{\underline{3 \textcircled{4}}} & \multicolumn{2}{c}{\underline{1 \textcircled{2}}} & (2) \\
      \underline{5} & \multicolumn{3}{c}{\underline{1 \textcircled{3} \textcircled{4}}} & \underline{2} & (2) \\
      \underline{5} & \multicolumn{4}{c}{\underline{1 2 \textcircled{3} \textcircled{4}}} & (2) \\
      \multicolumn{2}{c}{\underline{1 \textcircled{5}}} & \multicolumn{3}{c}{\underline{2 3 4}} & (1) \\
      \multicolumn{3}{c}{\underline{1 2 \textcircled{5}}} & \multicolumn{2}{c}{\underline{3 4}} & (1) \\
      \multicolumn{4}{c}{\underline{1 2 3 \textcircled{5}}} & \underline{4} & (1) \\
      \multicolumn{5}{c}{\underline{1 2 3 4 \textcircled{5}}} & (1) \\
      \hline
      \multicolumn{5}{c}{total inversions:} & 10
    \end{tabular}
    \end{center}
    Since this is just a merge-sort with some additional $O(1)$ steps in the merge phase, the complexity is $O(n\log n)$.

\end{enumerate}


\item

Let $A[1\ldots n]$ be an array of $n$ elements and $B[1\ldots m]$ another array of $m$ elements, with $m\leq n$. Note that neither $A$ nor $B$ is sorted. The problem is to compute the number of elements of $A$ that are smaller than $B[i]$ for each element $B[i]$ with $1\leq i\leq m$.
For simplicity, we assume that no two elements of $A$ are equal and no two elements of $B$ are equal.

For example, let $A$ be $\{30, 20, 100, 60, 90, 10, 40, 50, 80, 70\}$ of ten elements. Let $B$ be $\{60, 35, 73\}$ of three elements. Then, your answer should be the following: for 60, return 5 (because there 5 numbers in $A$ smaller than 60); for 35, return 3; for 73, return 7.

\begin{enumerate}
\item
Design an $O(n\log n)$ time algorithm for solving the problem. {\hfill \bf (5 points)}

\textit{Answer:} Sort $A$. This takes $O(n\log n)$ time. For each element $b$ in $B$, do a binary search for $b$ in $A$. This takes $O(m\log n)$ time. The index where $b$ is found in $A$ (or where $b$ would be found in $A$ if it is not present in $A$) is the number of elements in $A$ less than $b$. Complexity is $n\log n + m\log n = O(n\log n)$.

\item
Design an $O(nm)$ time algorithm for the problem. Note that this is better than the $O(n\log n)$ time algorithm if $m<\log n$. {\hfill \bf (5 points)}

\textit{Answer:} For each element $b$ in $B$, compare $b$ to each element of $A$ and keep a count of the number of elements of $A$ that are smaller than $b$. Since we iterate through $A$ $m$ times, complexity is $O(nm)$.

\item
Improve your algorithm to $O(n\log m)$ time. Because $m\leq n$, this is better than both the $O(n\log n)$ time and the $O(nm)$ time algorithms. {\hfill \bf (20 points)}


{\bf Hint:} Use the divide and conquer technique. Since $m\leq n$, you cannot sort the array $A$ because that would take $O(n\log n)$ time, which is not $O(n\log m)$ as $m$ may be much smaller than $n$.

\textit{Answer:} Sort $B$ from greatest to least. Create another array of integers, $C$, of size $m$. 
For each element of $A$, perform a binary search for that element in $B$ and return the index $i$ below where it is found 
(or below where it would be found if it is not present in $B$). Increment $C[i]$. Once this is done for every element of $A$,
the number of items of $A$ that are smaller than a given item of $B$, is given by a rolling sum of $C$. That is, after $B$ is sorted,
the solution for $B[m]$ is given by $C[m]$, the solution for $B[m-1]$ is given by $C[m-1] + C[m]$, the solution for $B[m-2]$ is given
by $C[m-2]+C[m-1]+C[m]$, etcetera. The end result of $B$ and $C$ for the example given in the problem statement is as shown below:

\begin{center}
\begin{tabular}{c|c|c|c|}
\cline{2-4}
B: & 73 & 60 & 35 \\
\cline{2-4}
\end{tabular} \\ \vspace{2mm}
\begin{tabular}{c|c|c|c|}
\cline{2-4}
C: & 2 & 2 & 3 \\
\cline{2-4}
\end{tabular}
\end{center}

Sorting $B$ is $O(m\log m)$. Performing a binary search on $B$ for every element in $A$ is $O(n\log m)$. Finding a given solution
from $C$ is at worst $O(m)$. Therefore this algorithm is $m\log m + n\log m + m = O(n\log m)$.
\end{enumerate}


{\bf Note:}
You will receive the full 30 points if you give an $O(n\log m)$ time algorithm directly for (c) without giving any algorithms for (a) or (b).


\item
{\bf (20 points)}
Solve the following recurrences (you may use any of the methods we studied in class). Make your bounds as small as possible (in the big-$O$ notation). For each recurrence, $T(n)$ is constant for $n\leq 2$.

\begin{enumerate}

\item
$T(n)=2\cdot T(\frac{n}{2})+ n^3$.

\item
$T(n)=4\cdot T(\frac{n}{2})+n\sqrt{n}$.

\item
$T(n)=2\cdot T(\frac{n}{2})+n\log n$.

\item
$T(n)=T(\frac{3}{4}\cdot n)+n$.
\end{enumerate}

\textit{Answer:} See attached paper.

\item
 {\bf (20 points)}
You are consulting for a small computation-intensive investment company, and
they have the following type of problem that they want to solve over and over.
A typical instance of the problem is the following. They are doing a simulation
in which they look at $n$ consecutive days of a given stock, at some point in the past.
Let's number the days $i=1,2,\ldots,n$; for each day $i$, they have a price $p(i)$ per share for the stock on that day. (We'll assume for simplicity that the price was fixed during each day.) Suppose during this time period, they wanted to buy 1000 shares on some day and sell all these shares on some (later) day. They want to know: When should they have bought and when should they have sold in order to have made as much money as possible? (If there was no way to make money during the $n$ days, you should report this instead.)

For example, suppose $n=5$, $p(1)=9$, $p(2)=1$, $p(3)=5$, $p(4)=4$, $p(5)=7$. Then you should return ``buy on $2$, sell on $5$'' (buying on day $2$ and selling on day $5$ means they would have made \$6 per share, the maximum possible for that period).

Clearly, there is a simple algorithm that takes time $O(n^2)$: try all possible pairs of buy/sell days and see which makes them the most money. Your investment friends were hoping for something a little better.

Design an algorithm to solve the problem in $O(n\log n)$ time. Your algorithm should use the divide-and-conquer technique.


{\bf Note:} The divide-and-conquer technique can actually solve the problem in $O(n)$ time. But such an algorithm is not required for this assignment. You may think about it if you would like to challenge yourself.

\textit{Answer:} Set the best pair of days $r$ to be the first day and the first day. Iterate through the list of day-price pairs,
tracking the current best day to buy on (e.g. the one with the lowest price) and the current best day to sell on 
(e.g. the one with the highest price). Anytime the best day to buy on is reset, set the best day to sell on to also be that day.
Anytime the best day to sell on is reset, check to see if the current best pair of days should be updated. That is, check to see if the
new best days to buy and sell result in greater profit than $r$. If not, carry on iterating. If so, update $r$.
Once the end of the list is reached, return $r$. If $r$ is a pair of the same day, then it is not possible to make money during the 
n days. Pseudo code for the algorithm is provided below:
\begin{verbatim}
// days is an array of Days
//    : Day.price is the price of the stock that day
//    : Day.id is the day - e.g. if Day.id is 1, then that day is day one
// if this function returns the same day on which to buy and sell, 
// then it is impossible to make money
buySell(days[1...n]) {

    // a day whose price is currently the best to buy on
    min = days[1]

    // a day whose price is currently the best to sell on, 
   	// initialized to a day whose price is 0
    max = Day.price(0)

    // a pair of days representing the best days to buy and sell
    bestDays = (days[1], days[1])

    for (day in days) {
        if (max.price - min.price 
            > bestDays.sellDay.price - bestDays.buyDay.price) {
            min = day          // update the min
            max = Day.price(0) // reset the max to a day with price 0
        }
        else if (day.price > max.price) {
            max = day // update the max
            if (updateBest(min, max, bestDays) {
                bestDays = (min, max) // update the best days
                max = Day.price(0)    // reset the max to a day with price 0
            }
        }
    }
    return bestDays
}
\end{verbatim}
All of the work is done in an iteration over the list of days, so this algorithm is $O(n)$.
\end{enumerate}

{\bf Total Points:} 100

\end{document}
